\documentclass[a4paper,12pt]{article} 

\usepackage{hyperref} 

\usepackage[inline]{enumitem}


% ******
% Layout
% ******

\usepackage[left=3cm, right=3cm, top=2.5cm, bottom=2.5cm]{geometry}
\setlength\parindent{0in}
\setlength{\headheight}{14.5pt}
%	Enlarged headheightbecause of a fancyhdr warning


% ******
% Tables
% ******

\usepackage{ltablex}
%	Defines tabularx environment which allows for a display break (in contrast to \usepackage{tabularx})
\keepXColumns
%	Necessary to keep the tables aligned to the left.
\setlength\LTpre{0in}
%\setlength\LTpost{0in}
%	Vertical space before and after tabularx

% ************
% Publications
% ************

\usepackage{biblatex}[bibstyle=authortitle]
\addbibresource{./bibliography/bibliography.bib}
\nocite{Hajek2018}
\defbibenvironment{midbib} % Suppress the numbering in bibliography
  {\list
     {}
     {\setlength{\leftmargin}{\bibhang}%
      \setlength{\itemindent}{-\leftmargin}%
      \setlength{\itemsep}{\bibitemsep}%
      \setlength{\parsep}{\bibparsep}}}
  {\endlist}
  {\item}


% *******************
% Headers and footers
% *******************

\usepackage{fancyhdr}
\pagestyle{fancy}
\lhead{P. H\'ajek, Academic CV}
\rhead{\today}
\cfoot{\thepage}






% *********
% Formating
% *********

\usepackage{libertine}

\usepackage{titlesec,xcolor}
\titleformat{\section}{\color{olive}\normalfont\scshape}{}{}{} 
%	Customization of sections headings.



\begin{document}


{\Large Pavel H\'ajek}\\[.2cm]
University of Augsburg\\
Universitätsstraße 14 \\
86 159 Augsburg, Germany \\[.2cm]
Phone: +49 176 995 71920\\
Email: \href{mailto:hajek_pavel@yahoo.de}{hajek\_pavel@yahoo.de}\\[.5cm]
Born on September 29, 1987 in Prague, Czech Republic\\
Nationality:  Czech

\section*{Current position}

\emph{PhD student} supervised by Prof. Dr. Kai Cieliebak at the Department for Analysis and Geometry, Faculty of Mathematics at the University of Augsburg.

\section*{Research interests}

\begin{description}[font=\normalfont]
\item[Actively:] \emph{$\mathrm{IBL}_\infty$-algebras, string topology, perturbative Chern-Simons theory.}
\item[Generally:] symplectic geometry, mathematical physics, dynamical systems.%(e.g., work of E. Witten and P. Mnev). 
\end{description}

\section*{Education}

\begingroup
\def\arraystretch{1.5}
\def\minipagewidthE{.846\textwidth}
\begin{tabularx}{\textwidth}{@{}lX@{}}
2014--now: & \begin{minipage}[t]{\minipagewidthE}
\emph{PhD in Mathematics, University of Augsburg} \\
Expected to finish in summer 2019. Thesis on $\mathrm{IBL}_\infty$-algebras, string topology and perturbative Chern-Simons theory. \end{minipage} \\
2011--2014: & \begin{minipage}[t]{\minipagewidthE}
\emph{MSc in Theoretical and Mathematical Physics, LMU Munich} \\ 
High distinction. Thesis on Eilenberg-Steenrod axioms for a homology theory based on manifolds with corners. 
\end{minipage}\\
2007--2011: & \begin{minipage}[t]{\minipagewidthE}
\emph{BSc in Physics, Charles University in Prag} \\ 
Distinction. Thesis on dynamical symmetries in classical and quantum mechanics.
\end{minipage} \\
2007--2010: &  \begin{minipage}[t]{\minipagewidthE} 
\emph{BSc in Mathematics, Charles University in Prag} \\ 
High distinction. Thesis on integrability of a generalization of the Lagrange top to higher dimensions.
\end{minipage}
\end{tabularx}
\endgroup

\section*{Scholarships}

\begingroup
\def\arraystretch{1.5}
\begin{tabularx}{\textwidth}{@{}lX@{}}
2012--2013: & Full scholarship for studying in Germany from the DAAD. \\
2007--2011: & Merit scholarship from the Charles University.
\end{tabularx}
\endgroup

\section*{Preprints}

\printbibliography[heading=none,env=midbib] 

\section*{Teaching}

\begingroup
\def\arraystretch{1.5}
\def\minipagewidthT{.837\textwidth}
\begin{tabularx}{\textwidth}{@{}lX@{}}
SS20 & Exercise class in Floer Theory.\\
WS19 & \begin{minipage}[t]{\minipagewidthT} 
\begin{itemize}[leftmargin=*,itemsep=-1ex]
\item 4 $\times$ exercise class in Mathematics for physicists I.
\item Exercise class in Symplectic Geometry.
\end{itemize}
\end{minipage} \\
WS16--SS17 & Teaching assistant in  Analysis I \& II. \\
SS16 & Coorganizer of a seminar on Floer homology. \\
WS14--SS15 & \begin{minipage}[t]{\minipagewidthT} 
\begin{itemize}[leftmargin=*,itemsep=-1ex]
\item Teaching assistant in Linear Algebra~I \& II.
\item Tutor in state exam preparatory courses for mathematics teachers.
\end{itemize}
\end{minipage}\\
SS12--WS13 & Tutor in Algebraic Topology~I~\&~II.	   		   
\end{tabularx}
\endgroup

\section*{Administrative experience}

\begingroup
\def\arraystretch{1.5}
\def\minpagewidthAE{.92\textwidth}
\begin{tabularx}{\textwidth}{@{}lX@{}}
2018 & \begin{minipage}[t]{\minpagewidthAE}
Workshop on Symplectic Field Theory IX, University of Augsburg, August 25--31 --- responsible for the \href{https://www.math.uni-augsburg.de/prof/geo/SFTIX/}{webpage}, registration, videos.
\end{minipage}
\end{tabularx}
\endgroup

\section*{Talks given}
\def\clap#1{\hbox to 0pt{\hss#1\hss}}
\begingroup
\def\arraystretch{1.5}
\setlength\LTpost{1ex}
\def\minpagewidthTG{.9207\textwidth}
\begin{tabularx}{\textwidth}{@{}lX@{}}
2019 & \begin{minipage}[t]{\minpagewidthTG}
\begin{itemize}[leftmargin=*,itemsep=-1ex]
\item Computations of the $\mathrm{IBL}_\infty$-structure, Workshop on String field theory, BV quantization, and moduli spaces, Simons Center for Geometry and Physics, Stony Brook, U.S.A., May 20--24.
\item  Explicit computation of Feynman integrals, Seminar for symplectic geometry, University of Augsburg, Augsburg, Germany, May 13, 2019.
\item $\mathrm{IBL}_\infty$-formality and Poincar\'e duality models, Seminar for symplectic and contact geometry at the University of Hamburg, Hamburg, Germany, April 25.
\item Chern-Simons theory and string topology, Seminar of the Research Institute for Mathematical Science, Kyoto, Japan, March 14.
\item Feynman integrals with the Green kernel, Seminar of the Mathematical Institute at the University of Potsdam, Potsdam, Germany, February 28.
\item $\mathrm{IBL}_\infty$-structure and string topology conjecture, 39th Winter School Geometry and Physics, Srn\'i, Czech Republic, January 12--19.
\end{itemize}
\end{minipage}\\
2016 & \begin{minipage}[t]{\minpagewidthTG}
Presentation of a part of the proof of the Cheeger-M\"uller theorem, Block seminar on Torsion in Geometry and Topology, Schloss Gollwitz, Brandenburg, Germany, July 3--8.
\end{minipage}\\
2015 & \begin{minipage}[t]{\minpagewidthTG}
Homology theory based on manifolds with corners,  Meeting of symplectic geometers, Weimar, Germany, 16--18 January. 
\end{minipage} \\
\end{tabularx}
\begin{description}[font=\normalfont,leftmargin=2.84em,rightmargin=0em]
\item[Local seminars at the University of Augsburg:]\strut\\
Chas-Sullivan string topology, Cyclic homology, Seiberg-Witten theory, symplectic capacities and ball packing problem, Witten's non-perturbative treatment of Chern-Simons theory, linking numbers and Green kernels, dynamics near the Lagrange points in the restricted three body problem, molecules of the Euler top and other topics.
\end{description}
\endgroup
\clearpage
%******************************
\section*{Conferences attended}
%******************************

\begingroup
\def\arraystretch{2}
%\hypersetup{hidelinks}
\def\minpagewidthCA{.92\textwidth}
\begin{tabularx}{\textwidth}{@{}lX@{}}

2019 & \begin{minipage}[t]{\minpagewidthCA}
\begin{itemize}[leftmargin=*,itemsep=-1ex]
\item \href{http://scgp.stonybrook.edu/archives/25214}{Workshop} on String field theory, BV quantization, and moduli spaces, Simons Center for Geometry and Physics, Stony Brook, U.S.A., May 20--24.
\item \href{http://conference.math.muni.cz/srni/}{39th} Winter School Geometry and Physics, Srn\'i, Czech Republic, January 12--19.
\end{itemize}
\end{minipage}\\

2018 & \begin{minipage}[t]{\minpagewidthCA} 
%\begin{itemize}[leftmargin=*]
%\item
\href{https://www.math.uni-augsburg.de/prof/geo/SFTIX/}{Workshop} on Symplectic Field Theory IX, University of Augsburg, Germany, August 25--31.
%\end{itemize}
\end{minipage}\\

2017 & \begin{minipage}[t]{\minpagewidthCA} 
%\begin{itemize}[leftmargin=*]
%\item
Meeting of symplectic geometers, Free University of Berlin, Germany, February 17--19.
%\end{itemize}
\end{minipage}  \\

2016 & \begin{minipage}[t]{\minpagewidthCA}
\begin{itemize}[leftmargin=*,itemsep=-1ex]
\item \href{https://www.math.uni-potsdam.de/professuren/geometrie/lehre/blockseminare/2016-brandenburg/}{Block} seminar on  Torsion in Geometry and Topology, Schloss Gollwitz, Brandenburg, Germany, July 3--8.
\item \href{https://www.math.uni-augsburg.de/prof/geo/Cast2016/}{X Workshop} on Symplectic Geometry, Contact Geometry, and Interactions, University of Augsburg, Germany, February 25--27. 
\end{itemize}\end{minipage}\\

2015 & \begin{minipage}[t]{\minpagewidthCA} 
\begin{itemize}[leftmargin=*,itemsep=-1ex]
\item \href{http://grk1670.math.uni-hamburg.de/ratstr2015/}{Summer} School on String Topology and Rational Homotopy Theory, University of Hamburg, Germany, September 2--4.

\item \href{https://lalondeteleman.weebly.com/main-conference.html}{Moduli} Spaces in Symplectic Topology and in Gauge Theory, CIRM, Marseille, France, June 1--5.

\item \href{http://conference.math.muni.cz/srni/}{35th} Winter School Geometry and Physics, Srn\'i, Czech Republic, 17--24 January.

\item  Meeting of symplectic geometers, Weimar, Germany, 16--18 January.
\end{itemize}
\end{minipage}\\
 
2014 & \begin{minipage}[t]{\minpagewidthCA} 
%\begin{itemize}[leftmargin=*]
%\item
\href{https://www.lebesgue.fr/content/sem2014-loops}{Loop} spaces in geometry and topology, University of Nantes, France,  1--5 September.
%\end{itemize}
\end{minipage}\\

2013 & \begin{minipage}[t]{\minpagewidthCA} 
%\begin{itemize}[leftmargin=*]
%\item
\href{https://www.uni-muenster.de/FB10/Service/show_article.shtml?id=4161\&brettid=8}{Minicourse} on free loop spaces in topology and physics, University of M\"unster, Germany, 24 April.
%\end{itemize}
\end{minipage}\\

2012 & \begin{minipage}[t]{\minpagewidthCA}
%\begin{itemize}[leftmargin=*]
%\item
\href{http://www.projects.science.uu.nl/poisson2012/Home.php}{Poisson} Geometry in Mathematics and Physics, University of Utrecht, Netherlands, 23 July--3 August.
%\end{itemize}
\end{minipage}
\end{tabularx}
\endgroup

%*******************
\section*{Languages}
%*******************

\begingroup
\def\arraystretch{1}
\setlength\LTpost{1ex}
\begin{tabularx}{\textwidth}{@{}lX@{}}
\emph{Czech}& mother tongue,\\
\emph{English} & fluent, \\
\emph{German} & fluent.
\end{tabularx}
Vocabulary to teach and discuss mathematics in all three languages.
\endgroup

\section*{Technical skills}

\begin{itemize}[leftmargin=*]
\item Programming in Wolfram~\textit{Mathematica}\textsuperscript{\textregistered} (computing Feynman integrals for spheres, searching for trajectories with prescribed itineraries in the three body problem) and Object Pascal.
\item Linux administrator skills.
% microchip, triangulation, minimization of logical function
\end{itemize}

%\section*{Webpages}
%
%https://github.com/p135246/

\section*{Referees}

\begin{description}[font=\normalfont]
\item[Prof.~Dr.~Kai Cieliebak] --- University of Augsburg, Universitätsstr.~14, 86159, Augsburg, Germany. Phone: 	+49 821 598 - 2138. Email: \href{mailto:Kai.Cieliebak@math.uni-augsburg.de}{Kai.Cieliebak@math.uni-augsburg.de}
\item[Prof.~Dr.~Urs Frauenfelder] --- University of Augsburg, Universitätsstr.~14, 86159, Augsburg, Germany. Phone: +49 821 598 - 2158.  Email: \href{mailto:Urs.Frauenfelder@math.uni-augsburg.de}{Urs.Frauenfelder@math.uni-augsburg.de}
\end{description}

%\section*{Technical skills}
%
%Operating systems: Linux -- administrator skills, Windows -- basic user
%
%advanced LaTeX user
%
%Programming: Projects in Object Pascal, Wolfram Mathematica, C and Assembler
%
%Basic understanding of PHP, Haskel, Prolog,
%


%
%\section*{Hobbies}
%
%Tennis, windsurfing and dancing.

%\section*{Hobbies}
%
%Windsurfing, tennis, dancing.

\end{document}
